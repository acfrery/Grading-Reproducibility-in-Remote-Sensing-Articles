\documentclass[journal]{IEEEtran}

\usepackage{cite}
\usepackage{texnames}
\usepackage{graphicx}
\graphicspath{{../Figures/PDF/}{}}
\DeclareGraphicsExtensions{.pdf,.jpeg,.png}

\usepackage{amsmath}
\interdisplaylinepenalty=2500
\usepackage{algorithmic}
\usepackage{array}
\usepackage[caption=false,font=footnotesize]{subfig}
%\usepackage{dblfloatfix}
%\usepackage[nomarkers]{endfloat}
\usepackage{url}
\usepackage{booktabs}

\usepackage{tikz}
\usetikzlibrary{shapes.geometric,arrows,trees}


\tikzstyle{root} = [rectangle, rounded corners, minimum width=2cm, minimum height=.7cm, text centered, draw=black, fill=blue!30]
\tikzstyle{files} = [rectangle, minimum width=1cm, minimum height=.5cm, text centered, draw=black, fill=white]
\tikzstyle{link} = [thick, -]

\begin{document}
	
\title{Reproducibility and Replicability in Remote Sensing Research}
	
\author{Alejandro~C.~Frery,~\IEEEmembership{Senior Member,~IEEE,}
Luis~Gomez,~\IEEEmembership{Senior Member,~IEEE,}
and~Qi~Wang,~\IEEEmembership{Senior Member,~IEEE}% <-this % stops a space
\thanks{A.\ C.\ Frery is with the \textit{Laborat\'orio de Computa\c c\~ao Cient\'ifica e An\'alise Num\'erica} -- LaCCAN, Universidade Federal de Alagoas, Macei\'o, Brazil, and with the Key Lab of Intelligent Perception and Image Understanding of the Ministry of Education, Xidian University, Xi'an, China. (email: acfrery@laccan.ufal.br)}% <-this % stops a space
\thanks{Luis Gomez is with the Universidad de Las Palmas de Gran Canaria, Spain}% <-this % stops a space
\thanks{Qi Wang is with the Northwestern Polytechnical University, China}% <-this % stops a space
\thanks{Manuscript received XX, accepted YY.}}
	
\markboth{IEEE Journal of Selected Topics on Applied Earth Observations and Remote Sensing,~Vol.~XX, No.~YY, Month~2020}%
{Frery \MakeLowercase{\textit{et al.}}: Grading Reproducibility}
	
\maketitle
	
\begin{abstract}
The abstract goes here.
\end{abstract}
	
\begin{IEEEkeywords}
Reproducibility,
Replicability,
Remote Sensing
\end{IEEEkeywords}
	

\IEEEpeerreviewmaketitle

\section{Introduction}\label{Sec:Introduction}
	
\IEEEPARstart{R}{eproducibility} is at the core of experimental sciences. 
It is also a basilar element of scientific integrity. 
The advent of data science is leading to new requirements and practices able to cope with the challenges posed by huge volumes of data, often of dynamic nature. 
	
Modern grounds for Reproducible Research were set before the widespread use of deep learning and other massively data-based techniques. 
	
Among the efforts made towards building Remote Sensing Reproducible Research environments, one may mention Code Ocean and GRSS Remote Sensing Code Library. 
These two initiatives belong to the IEEE scientific ecosystem. 
	
However, is this enough for healthy science in Remote Sensing? 
We do not think so. 
Many authors continuously strive to disseminate their research findings in such a way that any user will be able to validate them at a later stage. 
This includes the proposal of software architectures, the use of open data, FLOSS (Free/Libre Open Source Software), and other initiatives.

Barba~\cite{TerminologiesforReproducibleResearch} identifies several usages of the terms ``Reproducibility'' and ``Replicability,'' and also the emergence of the term ``Repeatability,'' among others.

We will discuss two core concepts: Reproducibility and Replicability, in order to simplify the discussion and to arrive to the suggestion of good scientific practices.

We will use the metaphor of dwarfs standing on the shoulders of giants which, in the words of Isaac Newton~\cite{LetterNewton}, is
\begin{quote}
If I have seen further it is by standing on the shoulders of Giants.
\end{quote}
Reproducibility consists in allowing the whole community reach your shoulders.
Replicability grants the community to stand on your shoulders.

A scientific work is reproducible if other researchers can obtain the data and the code and, effortlessly, obtain the same products (analyses and reports).
A scientific work is replicable if it reported in such a way that other researchers can perform similar studies and arrive to compatible conclusions.


According to Ref.~\cite{ReproducibilityandReplicabilityinScience2019}
		
Universally accessible and informative web page containing:
	\begin{enumerate}
		\item\label{item:ProjectID} Project identification (one project may host more than one paper; one paper may be hosted by more than one project)
		\begin{enumerate}
			\item Title
			\item Participants
			\item Summary
			\item Funding information
			\item Start date, state (in preparation, active, finished)
		\end{enumerate}
		\item Paper identification (if different from~\ref{item:ProjectID})
		\begin{enumerate}
			\item Title
: that of the corresponding author)
			\item Abstract
			\item PDFs of relevant versions, including information of its submission to repositories (arXiv, etc.), journal or conference
			\item\label{item:SourceDocumentFiles} \LaTeX\ and \BibTeX\ files, images, and plots
		\end{enumerate}
		\item\label{item:Platform} Computational platform (machine, model, operating system, software, libraries, and versions)
		\item Code (with comments)
		\item Data
		\item Which software components are FLOSS or free (including operating system)?
		\item Provide precise instructions and examples about:
		\begin{enumerate}
			\item How to install and run the code;
			\item How to read and modify the data;
			\item How to generate the plots, images (where they improved for visualization? how?), and tables (rounding, truncating, etc.)
		\end{enumerate}
	\end{enumerate}
	
	
	\begin{table*}[hbt]
		\centering
		\caption{Reproducibility scores of a research paper for the Remote Sensing community}
		\label{tab:my_label}
		\begin{tabular}{rccc}\toprule
			Question    & Answer & Score \\ \midrule
			1           & Does the project have a universally accessible web page & Yes & \\ 
			&  \\
			& \\ \bottomrule
		\end{tabular}
	\end{table*}

\section{Reproducibility, Samples and Simulation}

Many studies rely on the information provided by samples;
few of them detail the procedure with which they were collected.
A reproducible study should, besides providing the samples used, state the following:
\begin{enumerate}
	\item Objective criteria set \textit{a priori} for sample collection.
	\item Number of samples, sample size, and descriptive statistics for the aggregated data.
	\item Objective criteria for sample selection.
\end{enumerate}

Stochastic simulation is at the core of randomized algorithms as, for instance, Monte Carlo experiments.
In order to make such algorithms reproducible, apart from the information in Section~\ref{Sec:Introduction}, item~\ref{item:Platform}, the authors must inform the pseudo-random number generator and the seeds employed.

\section{How to Start Writing a Reproducible Article}

Starting well saves lots of time.
Here we make simple recommendations that may save time and efforts.
We assume that you use \LaTeX\ and \BibTeX.

The main idea consists in having a single repository for all the scientific texts you write: 
theses, 
reports, 
articles, 
letters,
reviews,
and miscellaneous documents.

Fig.~\ref{Fig:StructRepo} illustrates the recommended basic structure for a repository holding several \LaTeX\ files (or projects), along with their associated data and code.

\begin{figure}[hbt]
	\centering
	\includegraphics[width=.35\columnwidth]{DirectoryStructure}
	\caption{Recommended structure of a repository for scientific projects.}\label{Fig:StructRepo}
	%%% The source is in /Figures/DrawIO/DirectoryStructure.drawio
\end{figure}

Every directory may contain specific subdirectories.
For instance, \verb|Data| may contain \verb|CSV|, \verb|text|, and other directories with specific data files.

Notice that there is a single \BibTeX\ file (with extension \verb|.bib| in the \verb|Common| directory).
\BibTeX\ references can be split on several files, but these files should be common to all projects.
This avoids outdated and redundant bibliographic data bases.

Every article should be in its pwn directory.
Avoid using the name of the journal where you will submit your work for the directory and document names, as the destination may change along the process.

Data, figures, images and code should be common to all projects, as one typically reuses them.

Check Ref.~\cite{EditorialGRSL2015} for naming convention and revision handling of submitted manuscripts.


\section{Good Scientific Practices}

Science is not only made of positive results.
Sticking to this path may restrict the outcomes to confirmatory studies, avoiding those lines of research that do not produce immediately publishable results.
Scientific honesty requires telling the whole story, starting from clearly defining the research protocol~\cite{TellItlikeItIs}.
The research course may change along the work, but telling the whole story adds more to the scientific knowledge that reporting only the evidence and the conclusions that are aligned with the starting hypotheses.

\section{Recommendations}

Publishers should recognize those articles that comply with reproducibility criteria by assigning a badge.
According to Munaf\`o et al.~\cite{ManifestoReproducibleScience}, this practice increased by an order of magnitude articles with open data in the \textit{Psychological Science} journal.

Quoting Ref.~\cite{PraxisofReproducibleComputationalScience}:
\begin{quote}
a promise in a published paper
to make code and/or data ``available upon
request'' is not a reproducible practice: digital
artifacts should already be in a suitable repository.
\end{quote}
	
\section*{Acknowledgments}
The authors would like to thank...
	
\nocite{StatisticalAnalysesReproducibleResearch,%
RRComputationalHarmonicAnalysis,%
RREconometrics,%
RRSignalProcessing,%
AddressingNeedDataCodeSharingComputationalScience,%
ReproducibleResearchinComputationalScience,%
TenRulesReproducibleComputationalResearch,%
AddressingNeedDataCodeSharingComputationalScience,
SevenReasonsWhyaUsersGuidetoTransparencyandReproducibility,%
OutoftheBoxReproducibilityaSurveyofMachineLearningPlatforms,%
ReproducibilityofScientificResults2018,%
ReproducibleResearchandGIScienceanEvaluationUsingAGILEConferencePapers,%
TheStateofReproducibilityintheComputationalGeosciences}
	
\bibliographystyle{IEEEtran}
\bibliography{./Common/references}
	

	
\end{document}


